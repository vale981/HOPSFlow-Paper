\documentclass[reprint,aps,superscriptaddress]{revtex4-2}
\usepackage[unicode=true,bookmarks=true,bookmarksnumbered=false,bookmarksopen=false,breaklinks=false,pdfborder={0 0 1}, backref=false,colorlinks=true]{hyperref}
\usepackage{orcidlink}
\usepackage{microtype}
\usepackage{mathtools}
\usepackage{graphicx}
\usepackage{physics}
\usepackage{cleveref}
\usepackage{bm}
\bibliographystyle{apsrev4-2}


% HOPS/NMQSD
\def\sys{\ensuremath{\mathrm{S}}}
\def\bath{\ensuremath{\mathrm{B}}}
\def\inter{\ensuremath{\mathrm{I}}}
\def\nth{\ensuremath{^{(n)}}}

% unicode math
\iftutex
\usepackage{unicode-math}
\else
\usepackage{amssymb}
\def\z"{}
\def\UnicodeMathSymbol#1#2#3#4{%
 \ifnum#1>"A0
   \DeclareUnicodeCharacter{\z#1}{#2}%
  \fi}
\input{unicode-math-table}
\let\muprho\rho
\def\BbbR{\mathbb{R}}
\fi


\begin{document}
\preprint{APS/123-QED}
\title{Quantifying Energy Flow in Arbitrarily Modulated Open Quantum Systems}
\date{12.12.2100}

% fixme
\newcommand{\fixme}[1]{\marginpar{\tiny\textcolor{red}{#1}}}

\author{Valentin Boettcher\,\orcidlink{0000-0003-2361-7874}}
\affiliation{McGill University}
\altaffiliation[formerly at ]{TU Dresden}
\email{valentin.boettcher@mail.mcgill.ca}

\author{Konstantin Beyer\,\orcidlink{0000-0002-1864-4520}}
\email{konstantin.beyer@tu-dresden.de}
\affiliation{TU Dresden}

\author{Richard Hartmann\,\orcidlink{0000-0002-8967-6183}}
\email{richard.hartmann@tu-dresden.de}
\affiliation{TU Dresden}

\author{Walter T. Strunz\,\orcidlink{0000-0002-7806-3525}}
\email{walter.strunz@tu-dresden.de}
\affiliation{TU Dresden}




\begin{abstract}
\end{abstract}
\maketitle

\tableofcontents

\section{Introduction}
\label{sec:introduction}
The field of quantum thermodynamics has attracted much interest
recently~\cite{Talkner2020Oct,Rivas2019Oct,Riechers2021Apr,Vinjanampathy2016Oct,Binder2018,Kurizki2021Dec,Mukherjee2020Jan,Xu2022Mar}.
Quantum thermodynamics is, among other issues, concerned with
extending the standard phenomenological thermodynamic notions to
microscopic open systems.

The general type of model that is being investigated in this field is
given by the Hamiltonian
\begin{equation}
  \label{eq:4}
  H = H_{\sys} + ∑_{n} \qty[H_{\inter}^{(n)} + H_{\bath}^{(n)}],
\end{equation}
where \(H_{\sys}\) models a ``small'' system (from here on called
simply the \emph{system}) of arbitrary structure and the
\(H_{\bath}^{(n)}\) model the ``large'' bath systems with simple
structure but a large number of degrees of freedom. The
\(H_{I}^{(n)}\) acts on system and bath, mediating their interaction.

In this setting may make be possible to formulate rigorous microscopic
definitions of thermodynamic quantities such as internal energy, heat
and work that are consistent with the well-known laws of
thermodynamics. Currently, there is no consensus on this matter yet,
as is demonstrated by the plethora of proposals and discussions in
\cite{Rivas2019Oct,Talkner2020Oct,Motz2018Nov,Wiedmann2020Mar,Senior2020Feb,Kato2015Aug,Kato2016Dec,Strasberg2021Aug,Talkner2016Aug,Bera2021Feb,Bera2021Jun,Esposito2015Dec,Elouard2022Jul}.

This is particularly true for the general case where the coupling to
the baths may be arbitrarily strong. In this case the weak coupling
treatment that allows separate system and bath dynamics is not
applicable. Even the simple seeming question of how internal energy is
to be defined becomes non-trivial~\cite{Rivas2012,Binder2018} due to
the fact that \(\ev{H_{\inter}}\neq 0\).

In this way the bath degrees of freedom interesting in themselves,
which necessitates a treatment of the exact global unitary dynamics of
system and bath.

If no analytical solution for these dynamics is available, numerical
methods have to be relied upon. Notably there are perturbative methods
such as the Redfield equations for non-Markovian weak coupling
dynamics~\cite{Davidovic2020Sep} and also exact methods like the
Hierarchical Equations of Motion
HEOM~\cite{Tanimura1990Jun,Tang2015Dec}, multilayer
MCTDH~\cite{Wang2010May}, TEMPO~\cite{Strathearn2018Aug} and the
Hierarchy of Pure States HOPS~\cite{Suess2014Oct}\footnote{See
  \cite{RichardDiss} for a detailed account.}. Although the focus of
these methods is on the reduced system dynamics, exact treatments of
open systems can provide access to the global unitary evolution of the
system and the baths.

In this work we will focus on the framework of the ``Non-Markovian
Quantum State Diffusion'' (NMQSD)~\cite{Diosi1998Mar}, which is
briefly reviewed in~\cref{sec:nmqsd}. We will show in \cref{chap:flow}
that the NMQSD allows access to interaction and bath related
quantities. This novel application of the formalism constitutes the
main result of this work.

Based on the NMQSD and inspired by the ideas behind HEOM, a numerical
method, the ``Hierarchy of Pure States''
(HOPS)~\cite{RichardDiss,Hartmann2017Dec}, can be formulated. A brief
account of the method is given in \cref{sec:hops}.

The results of \cref{sec:flow}, most importantly the calculation of
bath and interaction energy expectation values, can be easily
implemented within this numerical framework. By doing so we will
elucidate the role of certain features inherent to the method. The
most general case we will be able to handle is a system coupled to
multiple baths of differing temperatures under arbitrary time
dependent modulation. As HOPS on its own is already a method with a
very broad range of applicability~\cite{RichardDiss}, we will find it
to be suitable for the exploration of thermodynamical settings.

In \cref{sec:applications} we apply this result to two simple systems.
As an elementary application, a brief study of the characteristics of
the energy flow out of a qubit into a zero temperature bath is
presented in \cref{sec:qubit-relax-char}. To demonstrate the current
capabilities of our method to the fullest we will turn to the
simulation of a quantum Otto-like
cycle~\cite{cite:Geva1992Feb,cite:Wiedmann2020Mar,cite:Wiedmann2021Jun}
in \cref{sec:quantum-otto-cycle}, which features a simultaneous time
dependence in both \(H_{\inter}\) and \(H_{\sys}\).

\section{Energy Flow with HOPS}
\label{sec:flow}

Let us proceed by briefly reviewing the fundamentals of the NMQSD and
the HOPS. A more thuro
\subsection{The NMQSD}
\label{sec:nmqsd}

\subsection{The HOPS}
\label{sec:hops}

\subsection{Bath Observables}
\label{sec:bath-observables}

\subsubsection{Bath Energy Change}
\label{sec:bath-energy-change}

\subsubsection{General Collective Bath Observables}
\label{sec:gener-coll-bath}


\section{Applications}
\label{sec:applications}

\subsection{Qubit Relaxation Characteristics}
\label{sec:qubit-relax-char}

\subsection{A Quantum Otto Cycle}
\label{sec:quantum-otto-cycle}



\begin{itemize}
\item see the chapter in my thesis
\item \textbf{Ask richard about phase transitions in spin boson}
\end{itemize}


\section{Outlook and Open Questions}
\label{sec:outl-open-quest}
\begin{itemize}
\item steady state methods
\item energy flow for portions of the bath -> adaptive method?
\end{itemize}

\bibliography{index}
\end{document}

%%% Local Variables:
%%% mode: latex
%%% TeX-master: t
%%% TeX-output-dir: "output"
%%% TeX-engine: luatex
%%% End:
